\begin{table}[H]
    \centering
    \begin{tabular}{|p{1.8cm}||p{1.8cm}|p{1.8cm}|p{1.8cm}|p{1.8cm}|p{2.3cm}|}
    \hline
    \textbf{Data Set} & \textbf{ILUC Face Recognition \%} & \textbf{Average Error (mm)} & \textbf{Optimise Success \%} & \textbf{Optimised Average Error (mm)} & \textbf{Improvement \%} \\
    \hline
    stationary & 100 \% & 0.699 & 38.8\% & 0.215 & 33.5\% \\
    \hline
    oscillating (XY) & 85.8\% & 0.806 & 34.4\% & 0.172 & 23.5\% \\
    \hline
    oscillating (XZ) & 87.7\% & 1.09 & 41.7\% & 0.238 & 28.3\% \\
    \hline
    \end{tabular}
    \caption{ILUC Recognition Accuracy. \textbf{ILUC Face Recognition \%} is the percentage of ILUC faces succesfully recognised. \textbf{Average Error (mm)} is the average error of the estimated ILUC faces. \textbf{Optimise Success \%} is the percentage of ILUC faces successfully optimised. \textbf{Optimised Average Error (mm)} is the average error of the optimised ILUC faces. \textbf{Improvement \%} is percentage of times the optimisation routine resulted in a smaller average error than the estimated case. Note, optimisation can introduce a larger error, if the initial estimate was already very close to the true value.}
    \label{table:iluc_recognition_accuracy}
    \end{table}