\section*{Code}
\begin{lstlisting}[language=C++, caption={Implementation of the new circle fit algorithm in the LUA software.}, label={lst:circle_fit}]
    Vector2f PipeFaceFinder::findCircleSVDHyper( 
        const Eigen::VectorXf& X, 
        const Eigen::VectorXf& Y, 
        float& radius )
    {
        Vector2f center;
        int      n = X.rows();
        VectorXf R = X.array() * X.array() + Y.array() * Y.array();
        MatrixXf D( n, 4 );
        D.col( 0 ) = R;
        D.col( 1 ) = X;
        D.col( 2 ) = Y;
        D.col( 3 ).setOnes();
    
        JacobiSVD<MatrixXf> svd( D, ComputeThinU | ComputeThinV );
        Vector4f            A;
        if ( svd.singularValues()( 3 ) / svd.singularValues()( 0 ) < 1e-12 ) // singular
        {
            A = svd.matrixV().col( 3 );
        }
        else // regular
        {
            DiagonalMatrix<float, 4> S( svd.singularValues() );
            Matrix4f                 N, W, Eigvecs;
            Vector4f                 mean, Eigvals, Astar;
    
            mean = D.colwise().mean();
            N << 8 * mean( 0 ), 4 * mean( 1 ), 4 * mean( 2 ), 2, //
                4 * mean( 1 ), 1, 0, 0,                          //
                4 * mean( 2 ), 0, 1, 0,                          //
                2, 0, 0, 0;
            W = svd.matrixV() * S * svd.matrixV().transpose();
    
            EigenSolver<MatrixXf> es;
            es.compute( W * N.inverse() * W );
            // taking the real part, since return type is complex
            Eigvals = es.eigenvalues().real();
            Eigvecs = es.eigenvectors().real();
    
            Vector4i index = Vector4i::LinSpaced( 0, 4 );
            std::stable_sort(
                index.begin(), 
                index.end(), 
                [&Eigvals]( size_t i1, size_t i2 ) { return Eigvals[i1] < Eigvals[i2]; } 
            );
    
            Astar = Eigvecs.col( index( 1 ) );
    
            A = W.ldlt().solve( Astar );
        }
    
        center = -A( Eigen::seq( 1, 2 ) ) / ( 2 * A( 0 ) );
    
        radius = std::sqrt( 
            A( 1 ) * A( 1 ) + A( 2 ) * A( 2 ) - 4 * A( 0 ) * A( 3 ) ) / ( 2 * std::abs( A( 0 ) ) 
        );
    
        return center;
    }
\end{lstlisting}

\begin{lstlisting}[language=C++, caption={Refactored pipe face finder.}, label={lst:find_circle_n_points}]
    pipeFace_s PipeFaceFinder::findCircleNPoints( const std::vector<Vector3f>& points,
                                              std::vector<float>&          pointsDisToPlane,
                                              std::vector<float>&          pointsDisToCircle )
{
    // ======================= //
    // FIND BEST FITTING PLANE //
    // ======================= //
    int nrOfPoints = points.size();

    // convert points to matrix
    MatrixXf pointsMatrix( nrOfPoints, 3 );
    pointsMatrix = Eigen::Map<const Eigen::Matrix<float, Eigen::Dynamic, 3, Eigen::RowMajor>>(
        reinterpret_cast<const float*>( points.data() ), nrOfPoints, 3 );

    // translate coordinates to centroid of data
    Vector3f centroid = calculateCentroid( pointsMatrix );
    pointsMatrix.rowwise() -= centroid.transpose();

    // plane fitting algorithm
    Vector3f normal;
    normal = fitPlaneNR( pointsMatrix );

    // ========================= //
    // PROJECT POINTS ONTO PLANE //
    // ========================= //
    MatrixXf projectedPoints( nrOfPoints, 3 );
    projectedPoints = projectPointsOntoPlane( pointsMatrix, normal );

    // calculate distance of points to the plane
    pointsDisToPlane.clear();
    MatrixXf pointsDisToPlaneVectors = projectedPoints - pointsMatrix;
    VectorXf pointsDisToPlaneVector  = pointsDisToPlaneVectors.rowwise().norm();

    // transform to plane coordinates
    MatrixXf planeBasis = spanPlane( normal );
    MatrixXf projectedPointsInPlaneCoordinates( nrOfPoints, 2 );
    projectedPointsInPlaneCoordinates = translateCoordinates( projectedPoints, planeBasis );

    // ============= //
    //  FIND CIRCLE  //
    // ============= //
    pipeFace_s tempPipeFace;

    // circle finding algorithm
    Vector2f circleCentre;
    if ( nrOfPoints == 3 )
    {
        circleCentre = findCircleLSTSQ( projectedPointsInPlaneCoordinates.col( 0 ),
                                        projectedPointsInPlaneCoordinates.col( 1 ),
                                        tempPipeFace.radius );
    }
    else
    {
        circleCentre = findCircleSVDHyper( projectedPointsInPlaneCoordinates.col( 0 ),
                                           projectedPointsInPlaneCoordinates.col( 1 ),
                                           tempPipeFace.radius );
    }

    // calculate distance to the circle
    pointsDisToCircle.clear();
    MatrixXf pointsDisToCirclecCentreVectors( nrOfPoints, 2 );
    pointsDisToCirclecCentreVectors  = 
        projectedPointsInPlaneCoordinates.rowwise() - circleCentre.transpose();
    VectorXf pointsDisToCircleVector = pointsDisToCirclecCentreVectors.rowwise().norm();
    pointsDisToCircleVector.array() -= tempPipeFace.radius;

    // ================================ //
    // TRANSFORM BACK TO THE VESSEL     //
    // ================================ //
    // translate back to lab frame
    Vector3f circleCentreGlobal =
        centroid + circleCentre( 0 ) * planeBasis.col( 0 ) 
        + circleCentre( 1 ) * planeBasis.col( 1 );

    // ============= //
    // STORE RESULTS //
    // ============= //
    std::copy( 
        circleCentreGlobal.data(), circleCentreGlobal.data() + 3, begin( tempPipeFace.centrePos ) 
    );
    std::copy( normal.data(), normal.data() + 3, begin( tempPipeFace.centreDirVec ) );
    std::copy( pointsDisToPlaneVector.data(),
               pointsDisToPlaneVector.data() + nrOfPoints,
               std::back_inserter( pointsDisToPlane ) );
    std::copy( pointsDisToCircleVector.data(),
               pointsDisToCircleVector.data() + nrOfPoints,
               std::back_inserter( pointsDisToCircle ) );

    tempPipeFace.found = true;

    return tempPipeFace;
}
\end{lstlisting}