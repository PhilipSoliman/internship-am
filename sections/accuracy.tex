\section{Accuracy of circle fits} \label{sec:accuracy}
This section gives a mathematical estimate for the accuracy of circle fit algorithms that minimize geometric distances. This is useful knowlegde for LUA, since this kind of `geometric circle fit' algorithm is used to form a 3-dimensional representation of the pipes, among other things (section \ref{ssec:code_flow}). This discussion then ends with the implementation in the LUA software of a new kind of circle fit algrorithm that has a variance that is the same as the least squares method, but no bias in its estimation of the circle radius. This section builds on the methods described in the paper by \citeauthor{alsharadqah_chernov_circle_fitting} \cite{alsharadqah_chernov_circle_fitting}.

\subsection{Statistical model}
In order to estimate the accuracy of a circle fit, we need to define a statistical model for the error in the input. To that end, we assume that the data points $(x_i, y_i)$ are noisy observations of the true points $(\tilde{x}_i, \tilde{y}_i)$, where the noise is i.i.d. normally distributed with bias 0 and variance $s^2$ \cite[section 2]{alsharadqah_chernov_circle_fitting}.
\begin{equation}
    x_i = \tilde{x}_i + \delta x_i, \quad y_i = \tilde{y}_i + \delta y_i, \quad \delta x_i, \delta y_i \sim N(0, s^2).
\end{equation}

This gives that the true points must lie on a circle, satisfying the equation
\begin{equation}
    (\tilde{x}_i - \tilde{a})^2 + (\tilde{y}_i - \tilde{b})^2 = \tilde{R}^2 \quad \text{for all } i, \text{ some midpoint } (\tilde{a}, \tilde{b}) \text{ and some radius } \tilde{R}.
\end{equation}

In the case of LUA the angles of the laser line scanners in the U-frame are known (section \ref{ssec:laser_line_scanners}). This suggests normalisation of the data points to the unit circle, which is done by
\begin{equation}
    u_i = \frac{\tilde{x}_i - \tilde{a}}{\tilde{R}} = \cos{\phi_i}, \quad v_i = \frac{\tilde{y}_i - \tilde{b}}{\tilde{R}} = \sin{\phi_i},
\end{equation}
where $\phi_i$ is the angle corresponding to the true point $(\tilde{x}_i, \tilde{y}_i)$ projected on the pipe (or ILUC ring/ -ribs) by the $i^{\text{th}}$ laser line scanner. This leads to the normalised data matrix $W$ with rows $W_i = (u_i, v_i, 1)^T$.

Define the vector of estimates $Q = (a, b, r)^T$ for the circle parameters. Then, the first order terms in the Taylor expansion of $Q$ in terms of $\delta x_i, \delta y_i$ is denoted by $\Delta_1Q$. The accuracy of the circle fit can then be estimated by the variance of $\Delta_1Q$ \cite[section 2]{alsharadqah_chernov_circle_fitting}.

\subsection{Accuracy estimation}
The geometric circle fit has both bias and (co-)variance. The bias is given by
\begin{equation}
    \text{E}[\Delta_1 Q] = \frac{2s^2}{\tilde{R}} \begin{pmatrix} 0 \\ 0 \\ 1 \end{pmatrix} \quad \text{\cite[equation 6.14]{alsharadqah_chernov_circle_fitting}},
\end{equation}
and therefore the bias exists only in the radius estimate as $\frac{2s^2}{\tilde{R}}$. The (co-)variance is given as
\begin{equation}
    \text{cov}(\Delta_1 Q) = s^2 (W^T W)^{-1} \quad \text{\cite[equation 8.12]{alsharadqah_chernov_circle_fitting}}.
\end{equation}

Taking the diagonal elements of the covariance matrix gives the variance of the estimates for the circle parameters. 