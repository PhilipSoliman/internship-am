\section{Accuracy of circle fits} \label{sec:accuracy}
This section gives a mathematical estimate for the accuracy of least squares circle fit algorithms that minimize geometric distances by solving the corresponding normal equations of the least squares. This is useful knowlegde for LUA, since this kind of `geometric circle fit' algorithm is used to form a 3-dimensional representation of the pipes, among other things (section \ref{ssec:code_flow}). This discussion then ends with the implementation in the LUA software of a new kind of circle fit algrorithm that has a variance that is the same as the geometric method, but possesses no bias in its estimation of the circle radius. This section builds on the methods described in the paper by \citeauthor{alsharadqah_chernov_circle_fitting} \cite{alsharadqah_chernov_circle_fitting}.

\subsection{Statistical model}\label{ssec:statistical_model}
In order to estimate the accuracy of a circle fit, we need to define a statistical model for the error in the input. To that end, we assume that the data points $(y_i, z_i)$ are noisy observations of the true points $(\tilde{y}_i, \tilde{z}_i)$, where the noise is i.i.d. normal random variables with bias 0 and variance $s^2$ \cite[section 2]{alsharadqah_chernov_circle_fitting} such that
\begin{equation}
    y_i = \tilde{y}_i + \delta y_i, \quad z_i = \tilde{z}_i + \delta z_i, \quad \delta y_i, \delta z_i \sim N(0, s^2).
\end{equation}

This gives that the true points must lie on a circle, satisfying the equation
\begin{equation}
    (\tilde{y}_i - \tilde{a})^2 + (\tilde{z}_i - \tilde{b})^2 = \tilde{R}^2 \quad \text{for } i = 1, \dots, n, \text{ midpoint } (\tilde{a}, \tilde{b}) \text{ and radius } \tilde{R}.
\end{equation}

In the case of LUA the angles of the laser line scanners in the U-frame are known (section \ref{ssec:laser_line_scanners}). This suggests normalisation of the data points to the unit circle, which is done by
\begin{equation}
    u_i = \frac{\tilde{y}_i - \tilde{a}}{\tilde{R}} = \cos{\phi_i}, \quad v_i = \frac{\tilde{z}_i - \tilde{b}}{\tilde{R}} = \sin{\phi_i},
\end{equation}
where $\phi_i$ is the angle corresponding to the true point $(\tilde{y}_i, \tilde{z}_i)$ projected on the pipe (or ILUC ring/ -ribs) by the $i^{\text{th}}$ laser line scanner. This leads to the normalised data matrix $W$ with rows $W_i = (u_i, v_i, 1)^T$.

Define the vector of estimates $Q = (a, b, r)^T$ for the circle parameters. Then, let the first order terms in the Taylor expansion of $Q$ in terms of $\delta y_i, \delta z_i$ be denoted by $\Delta_1Q$. The accuracy of the circle fit can then be estimated by the variance of $\Delta_1Q$ \cite[section 2]{alsharadqah_chernov_circle_fitting}.

\subsection{Accuracy estimation}
The geometric circle fit has both bias and (co-)variance. under the assumption that the number of data points approaches infinity $n \rightarrow \infty$, the bias is given by
\begin{equation}
    \text{E}[\Delta_1 Q] = \frac{2s^2}{\tilde{R}} \begin{pmatrix} 0 \\ 0 \\ 1 \end{pmatrix} \quad \text{\cite[equation 6.14]{alsharadqah_chernov_circle_fitting}},
    \label{eq:circle_fit_bias}
\end{equation}
and therefore the bias exists only in the radius estimate as $\text{E}[\Delta_1 R] =\frac{2s^2}{\tilde{R}}$. The covariance is given by
\begin{equation}
    \text{cov}[\Delta_1 Q] = s^2 (W^T W)^{-1} \quad \text{\cite[equation 8.12]{alsharadqah_chernov_circle_fitting}}.
    \label{eq:circle_fit_variance}
\end{equation}

Taking the diagonal elements of the covariance matrix gives the variance of the estimates for the circle parameters. Taking the square root of these variances gives the standard deviation $s(\cdot)$ of the estimates. The latter is good indicator of the accuracy of the circle fit.

It remains to estimate the noise level $s$ in the data and the true angles $\phi_i$ for all $i$. From section \ref{ssec:laser_line_scanners} it is known that the resolution of the laser line scanners is $47 \mu$m to $104 \mu$m in the $x$-direction and $2 \mu$m in the $z$-direction. This suggests that the noise level $s$ is on the order of $10^{-4}$ m. This is a rough estimate, but it is sufficient for the purposes of this discussion. As stated in the previous section \ref{ssec:statistical_model}, the true angles correspond to the angle of the laser line scanner in the U-frame.

Using the above circle fit accuracy can be calculated for the idealised scenario where a 24-inch (= 0.5 m) pipe is exactly in the center of the U-frame with zero pitch or yaw. In this case, the angles of the true poins are the same as the angles of the laser line scanners in equation \ref{eq:laser_angles}.

Substituting this into \cref{eq:circle_fit_bias,eq:circle_fit_variance} above gives
\begin{align*}
    \text{E}[\Delta_1 R] & = 0.04 \mu \text{m},                                       \\
    s[a]                 & = \sqrt{(\text{cov}[\Delta_1 Q])_{11}} = 0.05 \ \text{mm}, \\
    s[b]                 & = \sqrt{(\text{cov}[\Delta_1 Q])_{22}} = 0.09 \ \text{mm}, \\
    s[R]                 & = \sqrt{(\text{cov}[\Delta_1 Q])_{33}} = 0.05 \ \text{mm}.
\end{align*}

Hence, the accuracy of the geometric circle fit is given by
\begin{equation}
    \epsilon_{\text{midpoint}} \approx \sqrt{s[a]^2 + s[b]^2 } = 0.11 \ \text{mm}, \quad \epsilon_{\text{radius}} = s[R] \approx 0.05 \ \text{mm}.
    \label{eq:circle_fit_accuracy}
\end{equation}

Interestingly, the standard deviation of the midpoint estimate is larger in the $z$-direction than the $y$-direction. This can be attributed to the positioning of the laser line scanners in the U-frame; two on each side of the pipe near the $y$-plane and only one directly below in the $z$-plane. This suggests that the accuracy of the midpoint estimate can be improved upon by adding more laser line scanners in the $z$-direction.

\subsection{Implementation of a new circle fit algorithm}\label{ssec:circle_fit_implementation}
The function \lstinline[language=C]|PipeFaceFinder::findCircleNPoints| in the LUA software is used to fit pipes in 3D. First a plane is fitted to the set of bevels using a newton raphson method to minize the sum of squares. Subsequently, the points are projected onto the found plane and a circle is fitted to the projected points using a geometric circle fit algorithm of the type described above. Then the circle is projected back to 3D and the radius and midpoint are returned.

However above circle fit algorithm has a bias in the radius estimate. This is not ideal for the LUA system, since the radius of the pipe is a critical estimate in determining sufficient line-up (section \ref{ssec:oop_lua} and \ref{ssec:code_flow}). Therefore, a new circle fit algorithm was implemented in the LUA software that has the same variance as the least squares method, but no bias in the radius estimate. This is done by using the method of \citeauthor{alsharadqah_chernov_circle_fitting} termed `Hyper accurate algebraic circle fit' \cite{alsharadqah_chernov_circle_fitting}. 

Additionally, the old code was refactored to improve readability and maintainability. The hyper accurate fit is implemented in code listing \ref{lst:circle_fit} with the name \lstinline[language=C]|PipeFaceFinder::findCircleSVDHyper| and it is added to the function \lstinline[language=C]|PipeFaceFinder::findCircleNPoints| alongside the original geometric circle fit \lstinline[language=C]|PipeFaceFinder::findCircleLSTSQ| \ref{lst:find_circle_n_points}. 

Note that, since both the original geometric circle fit and the hyper accurate fit have the same variance, the accuracy of the circle fit with respect to the estimated midpoints is the same as in equation \ref{eq:circle_fit_accuracy}. Further investigation into the improvement of the radius estimate of the hyper accurate fit is yet to be done.